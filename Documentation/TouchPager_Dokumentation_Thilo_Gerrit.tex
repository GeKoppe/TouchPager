\documentclass[a4paper, 11pt]{scrartcl}

\usepackage[utf8]{inputenc}
\usepackage[ngerman]{babel}

\usepackage{mathptmx}
\renewcommand{\familydefault}{\sfdefault}
\usepackage[left=2.5cm, right=2.5cm, top=2.5cm, bottom=2cm]{geometry}
\usepackage[singlespacing]{setspace}

% \usepackage{hyperref}
\usepackage{amsmath}
\usepackage{amssymb}
\usepackage{graphicx}
\usepackage{xcolor}
\usepackage{floatflt,epsfig}
\usepackage{scrlayer-scrpage}
\usepackage{hyperref}
\usepackage{float}

\usepackage{xcolor}
\usepackage{floatflt,epsfig}
% \usepackage[fleqn]{amsmath}
\usepackage{listings}
\usepackage{color}
%\usepackage{minted}

\definecolor{dkgreen}{rgb}{0,0.6,0}
\definecolor{gray}{rgb}{0.5,0.5,0.5}
\definecolor{mauve}{rgb}{0.58,0,0.82}

% \lstset{frame=tb,
%   language=C,
%   aboveskip=3mm,
%   belowskip=3mm,
%   showstringspaces=false,
%   columns=flexible,
%   basicstyle={\small\ttfamily},
%   numbers=none,
%   numberstyle=\tiny\color{gray},
%   keywordstyle=\color{mauve},
%   commentstyle=\color{dkgreen},
%   stringstyle=\color{blue},
%   breaklines=true,
%   breakatwhitespace=true,
%   tabsize=3,
%   morekeywords={
    
%   }
%   %morecomment=[s][\color{dkgreen}]{//}
% }
\lstset{language=C++,
    basicstyle=\ttfamily,
    keywordstyle=\color{blue}\ttfamily,
    stringstyle=\color{red}\ttfamily,
    commentstyle=\color{green}\ttfamily,
    morecomment=[l][\color{magenta}]{\#}
    morekeywords={
        Messenger,Elegoo_TFTLCD,TouchScreen,VKeys % TODO hier müssen noch die restlichen Keywords eingepflegt werden
    }
}

\begin{document}
\thispagestyle{empty}
\ihead{
    \begin{footnotesize}
        Dokumentation Arduino Pager
    \end{footnotesize}
}
\chead{
    \begin{footnotesize}
        Lernfeld 7: Cyberphysische Systeme ergänzen
    \end{footnotesize}
}
\ohead{
    \begin{footnotesize}
        Thilo Drehlmann, Gerrit Koppe
    \end{footnotesize}
}
\vspace{0.2\textheight}
\begin{center}
    \begin{figure}[H]
        \begin{minipage}{0.3\textwidth}
            \includegraphics[scale=0.6]{Bilder/BBS}
        \end{minipage}
        \hspace{0.48\textwidth}
        \begin{minipage}{0.3\textwidth}
            \includegraphics[scale=0.6]{Bilder/sievers.png}
        \end{minipage}
    \end{figure}
    \vspace{1cm}
    \begin{Huge}
        \textbf{Dokumentation Arduino Pager} 
    \end{Huge}
    \\
    \vspace{0.1\textheight}
    \begin{Large}
        Autoren: Thilo Drehlmann, Gerrit Koppe
    \end{Large}
    \\
    \vspace{0.5cm}
    \begin{Large}
        Ausbildungsberuf: Fachinformatiker für Anwendungsentwicklung
    \end{Large}
    \\
    \vspace{0.5cm}
    \begin{Large}
        \today
    \end{Large}
\end{center}
\newpage
\thispagestyle{empty}
\tableofcontents
\newpage
\clearpage
\pagenumbering{arabic}
\section{Einleitung}
In dieser Dokumentation wird die Umsetzung eines bidirektionalen Pagers auf Basis der Arduino Plattform beschrieben. Zunächst werden Thema und Ziel des Projekts formuliert.
Anschließend werden wir auf die Planung der Ressourcen und des Ablaufs, sowie auf die benötigten Komponenten eingehen. Im Anschluss wird das Vorgehen während des Projektes dokumentiert
und abschließend das Ergebnis der Durchführung präsentiert.

\section{Formulierung des Themas}
\subsection{Beschreibung des Projektes}
Das Thema des Projektes ist es, eine bidirektionale Kommunikation zwischen zwei Geräten auf Arduino-Basis zu gewährleisten. Es soll die Möglichkeit bestehen,
Nachrichten zu verfassen, zu versenden und ebenso Nachrichten zu empfangen, die von einem anderen Arduino Gerät versendet wurden.

\subsection{Definition der Ziele, erwartetes Ergebnis}
Im Folgenden werden die allgemein Ziele des Projektes näher definiert.
\\
\\
1. Es soll möglich sein, mittels eines Touchscreens und einer virtuellen Tastatur, Zeichenketten auf einem, an den Arduino angeschlossenen Touchscreen, zu schreiben.
\\
2. Die eingegebenen Zeichenketten sollen, mittels Funkwellen, an ein anderes Gerät übertragen werden können.
\\
3. Das Gerät soll in der Lage sein, Funkwellen zu empfangen.
\\
4. Das Gerät soll außerdem in der Lage sein, die empfangenen Funkwellen wieder zu einer Zeichenkette zu übersetzen und auf einem Touchscreen anzuzeigen.
\\
5. Es soll ein graphisches User Interface auf dem Touchscreen geben.
\\
6. Es soll möglich sein, empfangene Nachrichten zwischenzuspeichern, damit neu empfangene Nachrichten nicht die vorherigen Nachrichten überschreiben.
\\
7. Es soll möglich sein, den Zwischenspeicher der Nachrichten über einen eigenen Menüpunkt abzurufen und die empfangenen Nachrichten zu verwalten.
\\
\\
Außerdem gibt es folgende, optionale Ziele:
\\
\\
a. Das User Interface soll farblich angepasst werden können.
\\
b. Es soll möglich sein, zu überprüfen, ob empfangsbereite Geräte in der Nähe sind.


\section{Ressourcen und Ablaufplanung}
\subsection{Ressourcen}

\subsubsection{Benötigte Hardware}
In Anhang Tabelle~\ref{tab:hardware} findet sich eine detaillierte, tabellarische Auflistung aller Komponenten, ihrer Aufgaben und ihrer Preise.
Alle Komponenten werden zwei mal benötigt, da eine Kommunikation zwischen zwei identischen Geräten hergestellt werden soll.

\subsubsection{Benötigte Software}
Zur Umsetzung des Projekts wird, um die Programmierung zu vereinfachen und den Quellcode schlanker zu halten, auf verschiedene externe Bibliotheken zurückgegriffen.
Eine detaillierte Auflistung dieser Bibliotheken findet sich im Anhang Tabelle~\ref{tab:software}.



\subsection{Planung der Umsetzung}


\subsubsection{Teilziele}
Folgende Teilziele wurden für das Projekt definiert:
\begin{enumerate}
    \item \textit{Vordefinierte Nachricht unidirektional übertragen}: Zunächst soll eine statisch eingestellte Nachricht zwischen zwei Arduino Mega mittels nRF24L01+ Transceiver
            Übertragen werden können, um zu prüfen, ob die Verbindung hergestellt werden kann.
    \item \textit{}
\end{enumerate}

\subsubsection{Erwartete Schwierigkeiten}
Im Folgenden werden alle Schwierigkeiten aufgelistet und erklärt, die während der Umsetzung des Projekts erwartet werden.
\begin{enumerate}
    \item \textit{Fehlersuche bei fehlerhafter Übertragung}: Da dieses Projekt darauf basiert, Funksignale zu versenden und zu empfangen und wir keine Gerätschaft
            besitzen, Funkwellen und Signalstärken dieser zu messen, wird es schwierig, den Fehler zu identifizieren, sollte eine Übertragung fehlschlagen.
    \item \textit{Distanzregulierung}: Die nRF24L01+ Transceiver können in verschiedenen Signalstärken senden, die programmatisch eingestellt werden müssen.
            Wird eine zu hohe Signalstärke konfiguriert, leidet darunter allerdings die Übertragungsqualität bei niedrigen Distanzen. Hier muss ein gutes Mittelmaß
            gefunden werden.
    \item \textit{Wechsel zwischen Empfang und Senden}: Da die Nachrichten bidirektional versendet werden sollen, die nRF24L01+ Transceiver aber nur halbduplex arbeiten,
            müssen wir einen rechtzeitigen Wechsel der Antenne zwischen Senden und Empfang garantieren. Sollten beide Geräte gleichzeitig Senden, werden beide Nachrichten
            verloren gehen.
    \item \textit{Empfang garantieren}: Da die Möglichkeit bestehen soll, gleich 
    \item \textit{Kein Multithreading}: Da Arduinos nicht Multithreading-fähig\footnote{vgl. Glossar~\ref{def:multithreading}: Multithreading} sind
\end{enumerate}

\subsubsection{Zeitliche Planung}


\section{Durchführung}


\section{Projektergebnis}

\newpage
\section{Anlagen}
\begin{small}

\begin{table}[H]
    \caption{Benötigte Hardware}\label{tab:hardware}
    \begin{tabular}{|p{0.3\textwidth}|p{0.5\textwidth}|p{0.1\textwidth}|}
        \hline
        \textbf{Hardware} & \textbf{Aufgabe} & \textbf{Kosten}
        \\
        \hline\hline
        Arduino Mega 2560 
        & 
        \begin{itemize}
            \item Zentrale Schnittstelle aller Komponenten
            \item Verwaltung der Logik / Programmierbarkeit
            \item 
        \end{itemize} 
        & 
        21,99€
        \\
        \hline
        Elegoo Uno TFT Touchscreen 2,8''
        &
        \begin{itemize}
            \item Anzeige von Nachrichten
            \item Eingabe von Nachrichten
            \item User Interface
        \end{itemize}
        &
        19,99€
        \\
        \hline
        nRF24L01+ Wireless Transceiver Modul
        &
        \begin{itemize}
            \item Nachrichten Übertragen und Empfangen
            \item Überprüfung Verfügbarkeit anderer Geräte
        \end{itemize}
        &
        5,-€
        \\
        \hline
    \end{tabular}
\end{table}

\begin{table}[H]
    \caption{Benötigte Software}\label{tab:software}
    \begin{tabular}{|p{0.2\textwidth}|p{0.5\textwidth}|p{0.2\textwidth}|}
        \hline
        \textbf{Bibliothek} & \textbf{Aufgabe} & \textbf{Quelle}
        \\
        \hline\hline
        Elegoo\text{\_}GFX.h 
        & 
        Kern Grafikbibliothek des Elegoo Uno TFT Touchscreens. Ermöglicht das Drucken von Zeichen / Formen auf TFT Display.
        &
        Mitgeliefert auf CD bei TFT Touchscreen
        \\
        \hline
        Elegoo\text{\_}TFTLCD.h
        &
        Hardware-Bibliothek des Elegoo Uno TFT Touchscreens. Verantwortlich für die Kommunikation des Programms mit der Hardware.
        &
        Mitgeliefert auf CD bei TFT Touchscreen
        \\
        \hline
        TouchScreen.h
        &
        Bibliothek des Touchscreens des Elegoo Uno TFT Touchscreens. Erlaubt das erkennen von Berührungen des Touchscreens und die Lokalisierung der
        Berührung.
        &
        Mitgeliefert auf CD bei TFT Touchscreen
        \\
        \hline
        SPI.h
        &
        Erlaubt die Kommunikation des Programms mit dem SPI Bus des Arduino Board
        &
        In Arduino IDE inkludiert
        \\
        \hline
        nRF24L01.h
        &
        Hardware Bibliothek der nRF24L01+ Transceiver. Erlaubt Kommunikation des Moduls mit dem Arduino Board
        &
        Github %TODO die Quellen ALLE noch mal checken und anpassen
        \\
        \hline
        RF24.h
        &
        Programmierbare Schnittstelle der nRF24L01+ Transceiver. 
        &
        Github
        \\
        \hline
        Arduino.h
        &
        Liefert Kernfunktionen der Arduino Boards.
        &
        In Arduino IDE inkludiert
        \\
        \hline
    \end{tabular}
\end{table}

\end{small}
\newpage
\section{Glossar}
\subsection{Technische Begriffe}
\textbf{Multithreading\label{def:multithreading}}
\\
Unter Multithreading versteht man in der Informatik den Prozess, ein Programm in mehrere Teilstränge aufzuteilen, die parallel 
ausgeführt werden.\footnote{Vgl. Quelle~\ref{itm:multithread}}

\newpage
\section{Quellenverzeichnis}
\subsection{Internetquellen}
\begin{enumerate}
    \item Storage Insider: Was ist Multithreading - Online unter \url{https://www.storage-insider.de/was-ist-multithreading-a-1017586/} $\left[\text{07.01.2023}\right]$ \label{itm:multithread}
\end{enumerate}


\end{document}