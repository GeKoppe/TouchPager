\documentclass[a4paper, 11pt]{scrartcl}

\usepackage[utf8]{inputenc}
\usepackage[ngerman]{babel}

\usepackage{mathptmx}
\renewcommand{\familydefault}{\sfdefault}
\usepackage[left=2.5cm, right=2.5cm, top=2.5cm, bottom=2cm]{geometry}
\usepackage[singlespacing]{setspace}

\usepackage{hyperref}
\usepackage{amsmath}
\usepackage{amssymb}
\usepackage{graphicx}
\usepackage{xcolor}
\usepackage{floatflt,epsfig}
\usepackage{scrlayer-scrpage}
\usepackage{hyperref}
\usepackage{float}

\usepackage{xcolor}
\usepackage{floatflt,epsfig}
% \usepackage[fleqn]{amsmath}
\usepackage{listings}
\usepackage{color}
%\usepackage{minted}

\definecolor{dkgreen}{rgb}{0,0.6,0}
\definecolor{gray}{rgb}{0.5,0.5,0.5}
\definecolor{mauve}{rgb}{0.58,0,0.82}

% \lstset{frame=tb,
%   language=C,
%   aboveskip=3mm,
%   belowskip=3mm,
%   showstringspaces=false,
%   columns=flexible,
%   basicstyle={\small\ttfamily},
%   numbers=none,
%   numberstyle=\tiny\color{gray},
%   keywordstyle=\color{mauve},
%   commentstyle=\color{dkgreen},
%   stringstyle=\color{blue},
%   breaklines=true,
%   breakatwhitespace=true,
%   tabsize=3,
%   morekeywords={
    
%   }
%   %morecomment=[s][\color{dkgreen}]{//}
% }
\lstset{language=C++,
    basicstyle=\ttfamily,
    keywordstyle=\color{blue}\ttfamily,
    stringstyle=\color{red}\ttfamily,
    commentstyle=\color{green}\ttfamily,
    morecomment=[l][\color{magenta}]{\#}
    morekeywords={
        Messenger,Elegoo_TFTLCD,TouchScreen,VKeys % TODO hier müssen noch die restlichen Keywords eingepflegt werden
    }
}

\begin{document}
\thispagestyle{empty}
\ihead{
    \begin{footnotesize}
        Dokumentation Arduino Pager
    \end{footnotesize}
}
\chead{
    \begin{footnotesize}
        Lernfeld 7: Cyberphysische Systeme ergänzen
    \end{footnotesize}
}
\ohead{
    \begin{footnotesize}
        Thilo Drehlmann, Gerrit Koppe
    \end{footnotesize}
}
\vspace{0.2\textheight}
\begin{center}
    \begin{figure}[H]
        \begin{minipage}{0.3\textwidth}
            \includegraphics[scale=0.6]{Bilder/BBS}
        \end{minipage}
        \hspace{0.48\textwidth}
        \begin{minipage}{0.3\textwidth}
            \includegraphics[scale=0.6]{Bilder/sievers.png}
        \end{minipage}
    \end{figure}
    \vspace{1cm}
    \begin{Huge}
        \textbf{Dokumentation Arduino Pager} 
    \end{Huge}
    \\
    \vspace{0.1\textheight}
    \begin{Large}
        Autoren: Thilo Drehlmann, Gerrit Koppe
    \end{Large}
    \\
    \vspace{0.5cm}
    \begin{Large}
        Ausbildungsberuf: Fachinformatiker für Anwendungsentwicklung
    \end{Large}
    \\
    \vspace{0.5cm}
    \begin{Large}
        \today
    \end{Large}
\end{center}
\newpage
\thispagestyle{empty}
\tableofcontents
\newpage
\clearpage
\pagenumbering{arabic}
\section{Einleitung}
In dieser Dokumentation wird die Konzeptionierung, Umsetzung, sowie 

\section{Formulierung des Themas}

\section{Ressourcen und Ablaufplanung}

\subsection{Benötigte Hardware}
In Anhang Tabelle~\ref{tab:hardware} findet sich eine detaillierte, tabellarische Auflistung aller Komponenten, ihrer Aufgaben und ihrer Preise.
Alle Komponenten werden zwei mal benötigt, da eine Kommunikation zwischen zwei identischen Geräten hergestellt werden soll.




\subsection{Planung der Umsetzung}


\subsubsection{Teilziele}
Folgende Teilziele wurden für das Projekt definiert:
\begin{enumerate}
    \item \textit{Vordefinierte Nachricht unidirektional übertragen}: Zunächst soll eine statisch eingestellte Nachricht zwischen zwei Arduino Mega mittels nRF24L01+ Transceiver
            Übertragen werden können, um zu prüfen, ob die Verbindung hergestellt werden kann.
    \item \textit{}
\end{enumerate}

\subsubsection{Erwartete Schwierigkeiten}
Im Folgenden werden alle Schwierigkeiten aufgelistet und erklärt, die während der Umsetzung des Projekts erwartet werden.
\begin{enumerate}
    \item \textit{Fehlersuche bei fehlerhafter Übertragung}: Da dieses Projekt darauf basiert, Funksignale zu versenden und zu empfangen und wir keine Gerätschaft
            besitzen, Funkwellen und Signalstärken dieser zu messen, wird es schwierig, den Fehler zu identifizieren, sollte eine Übertragung fehlschlagen.
    \item \textit{Distanzregulierung}: Die nRF24L01+ Transceiver können in verschiedenen Signalstärken senden, die programmatisch eingestellt werden müssen.
            Wird eine zu hohe Signalstärke konfiguriert, leidet darunter allerdings die Übertragungsqualität bei niedrigen Distanzen. Hier muss ein gutes Mittelmaß
            gefunden werden.
    \item \textit{Wechsel zwischen Empfang und Senden}: Da die Nachrichten bidirektional versendet werden sollen, die nRF24L01+ Transceiver aber nur halbduplex arbeiten,
            müssen wir einen rechtzeitigen Wechsel der Antenne zwischen Senden und Empfang garantieren. Sollten beide Geräte gleichzeitig Senden, werden beide Nachrichten
            verloren gehen.
    \item \textit{Empfang garantieren}: Da die Möglichkeit bestehen soll, gleich 
\end{enumerate}

\section{Durchführung}

\section{Projektergebnis}

\newpage
\section{Anlagen}


\begin{table}[H]
    \caption{Benötigte Hardware}\label{tab:hardware}
    \begin{tabular}{|p{0.4\textwidth}|p{0.4\textwidth}|p{0.1\textwidth}|}
        \hline
        \textbf{Hardware} & \textbf{Aufgabe} & \textbf{Kosten}
        \\
        \hline\hline
        Arduino Mega 2560 
        & 
        \begin{itemize}
            \item Zentrale Schnittstelle aller Komponenten
            \item Verwaltung der Logik / Programmierbarkeit
            \item 
        \end{itemize} 
        & 
        21,99€
        \\
        \hline
        Elegoo Uno TFT Touchscreen 2,8''
        &
        \begin{itemize}
            \item Anzeige von Nachrichten
            \item Eingabe von Nachrichten
            \item User Interface
        \end{itemize}
        &
        19,99€
        \\
        \hline
        nRF24L01+ Wireless Transceiver Modul
        &
        \begin{itemize}
            \item Nachrichten Übertragen und Empfangen
            \item Überprüfung Verfügbarkeit anderer Geräte
        \end{itemize}
        &
        5,-€
        \\
        \hline
    \end{tabular}
\end{table}


\newpage
\section{Glossar}
\section{Quellenverzeichnis}


\end{document}